%!TEX root = Thesis.tex

The comments from the committees in the defense are list as follows. We answer these questions and also make improvement in the corresponding manuscript.
\begin{enumerate}
  \item In Chapter~\ref{chap:PAGE}, to search the performance space, power and area should be always concerned as the minimized criteria. \\
  {\bf Answer:} Originally, power consumption is one of the performance metrics to explore. However it is practical to only minimize the performance dissipation and area for chip in the advanced technology node. We add these criteria as constraints in the optimization equation in the future works. 
  \item In Chapter~\ref{chap:PAGE}, since process corners are represented as the extremes of the technology node, these corners are also adoptable as the performance space to explore. \\
  {\bf Answer:} It is considerable that the process corners are part of the performance metrics in the exploration stage. In our future works, we will focus on the the different corners as exploration to achieve the practical design in the advanced technology.
  \item Is PAGE only applicable on design migration? If not, the dissertation should mention it. \\
  {\bf Answer:} Parallel genetic algorithm for performance exploration is not restricted to design migration. It is a general analog circuit sizing problem to explore the performance space of the targeting technology node. We also deliver this idea in Section~\ref{sec:PAGEIntro}. 
  \item Topology selection is not part of the contributions in this dissertation, it should not be mentioned in the dissertation. \\
  {\bf Answer:} In Chapter~\ref{chap:PAGE}, other than the topology selection, the major contribution to the analog circuit automation is the improvement of circuit sizing. We also emphasize the target in Section~\ref{subsec:PAGEContribute}
  \item The circuit level variable to performance space transformation is much easier than the inverse-direction. To the same performance metric, there are more than one sets of corresponding device variables. What is the policy for the authors to select? \\
  {\bf Answer: }In Section~\ref{sec:reTarg}, our approach uses a set of posynomial forms as a convex optimization to re-target the corresponding device-level variables. The selected variables are optimized by the convex optimization tool. 
  \item The initial solution of the parallel genetic algorithm changes a lot, how do the authors to avoid the instability.\\
  {\bf Answer:} Indeed, the initial solution effects the evolutionary characteristic in genetic algorithm. In current approach, we experiment our parallel genetic algorithm with at least 10 times to obtain the average result of each exploration. In the future work, since the existing netlist is accompanied with a set of design variables and corresponding performance metric. Therefore, it is possible to be an effective initial solution. We also introduce this goal as our future work in Chapter~\ref{chap:CFW}
  \item Why PAGE needs to do the parallel genetic algorithm for performance space searching first and then do simulated annealing? Could PGA achieve all of the performance searching?\\
  {\bf Answer:} According to \cite{PerfMap_ISQED2011}, although the explored performance metric represents the optimal in performance space, any high-level or non-ideal effect is not overlooked in the exploration process. Therefore, by employing global optimization such as ProSA, the entire circuit is re-examined into circuit simulator with real models and parameters from the advanced technology node. The explored population of performance metrics are the initial distributions, which is a converged space for ProSA to swift. We also improve our manuscript in Chapter~\ref{chap:PAGE} with this explanation.
  \item It is too vague to see the difference between traditional router and the proposed constraint-driven router, the authors should emphasize that. \\
  {\bf Answer:} According to Section~\ref{subsec:CCDARContribute}, Figure~\ref{fig:LayoutFlow} compare the traditional routing and our approach. The traditional flow manipulates the router with fixed and self-defined constraints. However it is not flexible with different analog designs. Our approach is able to extract the existing layout constraints as guidelines for the routers. Therefore, we compare the routing results among the commercial router with and without the constraints in Section~\ref{sec:CUCLMExp}. We also refine our manuscript in Chapter~\ref{chap:Intro}. 
  \item Why the number of VIAs in Flatten-route, HCDR-LE and HCDR-PD much more than the manual route? \\
  {\bf Answer:} First of all, due to the parasitic effects caused by VIA, the more bend in an individual wire, the more mismatches occur. Therefore, the manual-route tries to minimize the opportunities to change layer. In contrast, the commercial router minimizes the wirelength by changing layers frequently. It raises the number of VIAs in Flatten-route, HCDR-LE and HCDR-PD. We discuss the comparison more in Section~\ref{sec:CUCLMExp}, and then put the enhancement to reduce the number of VIAs as another constraints in our future work.
  \item The DRC errors in HCDR-PD are still more than 0. To the expectation from the designers, the DRC error more than 0 still means the uncompleted design. It is possible that these DRC error are the critical error which takes much more time to fix it. \\
  {\bf Answer:} In Section~\ref{sec:CUCLMExp}, our experiments compare 4 different routing behaviors with and without constraints. As Table~\ref{table:HCDRResult} illustrated, the number of DRC errors are more than 0 except manual-route. The idea to illustrate the difference number of DRC errors is the numbers represent the effort of finalize the layout without DRC error. However, the design without 0-DRC-error means uncompleted design. Due to the reason that the layout design should be as compact as possible. To fix any DRC error deep in a compact design takes effort to shift blocks and wires. We refine the discussion in Section~\ref{sec:CUCLMExp} and add the improvement of 0-DRC-error as our future work in Chapter~\ref{chap:CFW}
  \item Why the authors use CDT for layout preservation? It seems that the authors does not expend all the advantage of CDT as layout preservation. \\
  {\bf Answer:} We improve our manuscript with the explanation to apply CDT as our routing preservation mechanism in Section~\ref{subsec:RtPres} and Section~\ref{subsec:ReviewCDT}. Moreover, Constrained-Delaunay-Triangulation has not exercised as a layout preservation strategy to the fullest. We put it as our future enhancement. 
  \item The third experiment represents the different placement in the migrated layout. However, some of the original symmetry constraints are broken. What is the criteria for the topology generation to release the constraints? \\
  {\bf Answer:} In Section~\ref{subsec:ExpLDO}, our approach generates different topologies as layout solutions. Since the major constraint in this case is the area, part of the original layout constraints are released such as the symmetry group. However, it benefits the area obviously. We also improve the paragraph in Section~\ref{subsec:ExpLDO} with such discussion. 
  \item The area report in Table~\ref{table:HierProto}  of \cite{msc-bhattacharya-tcad06}+\cite{Chin_DMR_ICCAD2013} of umc90nm technology node has typo. \\
  {\bf Answer:} The area should be 17136($153\times 112$) and we revised it.
  \item The authors should emphasize the motivation to propose the flow of migration among each approach. \\
  {\bf Answer:} In Section~\ref{sec:contribution}, we claim our contributions to the analog design automation field with our design migration framework. However, we did not mention the motivation to use design migration instead of other approaches. In previous works, methodologies with non-template-based \cite{apnsi-pohung-dac07,GeneralRouter-CompEuro89,sensitAR-iccad90,arearouting-tcad1993,KOAN_ANAGRAMII-JSSC1991,AnalogRouteMatching-iccad2009,ppraic_Linfu_iccad2010,phLin-dac2008} implement layout generation without existing layout references, which bring the experience from the expert layout designer for generating layout on advanced technology nodes. Methodologies with template-based \cite{msc-bhattacharya-tcad06,ALP_YPWeng_iccad2011,ymYang-isqed2010,LAYGENII_TCAD13} reference the existing layout topologies but have not addressed the relationship among placement and routing. Otherwise, to the best of our knowledge, the design migration for analog circuit only mention the layout migration strategies except the synthesis stage currently. Our approaches first propose the framework to generate migrated layout from circuit to layout generation. We also improve our manuscript with the motivation in Section~\ref{sec:contribution}

\end{enumerate}