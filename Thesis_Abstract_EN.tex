% !TEX root = Thesis.tex

Analog design automation has become one of the most important issue to efficiently achieve design specification in integrated circuit industry. To elaborate this issue, constraint-driven design generation plays a key role in analog circuit synthesis and layout generation. During circuit synthesis, device models, performance specification and parasitics' effects consolidate the constraints for exploring the design equation. Yet, to figure out the set of non-linear design equation is complicated and time-consuming. Meanwhile, in layout generation stage, the layout dependent effect (LDE) should be taken into consideration since the performance of analog design is sensitive to physical layout constraints. Traditional layout strategy with manual manipulation requires tremendous time to satisfy specification for sign-off. Moreover, as advanced technology node develops, the complexity of device models and layout constraints raise as well. The hardship of realization analog circuit on advanced technology node can considerably drop off if the constraints corresponding to design are treated well during synthesis and layout generation.

In this dissertation, 3 stage of design automation strategies are proposed, parallel genetic performance exploration for circuit sizing, unified constraint-driven routing for analog layout and rapid prototyping for analog layout migration. Parallel genetic performance exploration analyze the limitation of a specific technology via genetic algorithm efficiently. Other than examine the corners of device model carefully, genetic algorithm provides a sketch of performance space to optimize. Experimental results show that the integration of our genetic performance exploration and probabilistic perturbation achieves both efficiency and accuracy on a radio-frequency distributed amplifier (RFDA) and a folded cascode operational amplifier (Op-Amp) in three different technologies.

The second approach, unified constraint-driven routing, provides an integration concept of generating constraints for industrial analog design. Most of the layout constraints provided by manufacturing foundries are technology-oriented. However, the design constraints provided by designers are critical as well. We proposes an unification of technology and design constraints to perform layout generation. By practicing on an analog functional block of tsmc 40nm SoC design which guarantees to be legalized and satisfies required analog constraints by DRC/LVS and post-layout simulation respectively, the results in wire matching for signal integrity show that the different routing priority generated by our approach can have significant performance impact.

In our last approach of this dissertation, a rapid design migration for analog circuit is presented. Layout generation for analog design in advanced CMOS technology is challenging by growing layout constraints and performance specifications. The manufacturing reliability and parasitic effects make the situation more intricate. To facilitate the utility of template-based analog layout generation, our migration approach extracts both placement and routing from an existing layout and then implements rapid prototyping into multiple results. In addition, the wires in the resulting layout are optimized for better performance. The experimental results demonstrate the possibility on multiple layout migration, such that more than 75\% routing of migrated layout is generated with qualified performance. 



%both nonuniversal and universal approaches that improve the performance of short-length Luby Transform (LT) codes with applications to multimedia communication. The key idea is to increase the expected \emph{output ripple size} to prevent Belief-Propagation (BP) decoding process from terminating prematurely. The nonuniversal approach with Soliton-based distribution and randomness-limited encoding scheme is discussed first. Then the universal approach is investigated focusing on constructing degree distributions with respect to the output ripple size.


%, where the output ripple size is the number of degree-$1$ output symbols


%LT codes have been proved to be capacity-achieving on erasure channels. The performance of these codes is dominated by the code length and the design of the degree distribution. The state of the art robust Soliton distribution (RSD) is designed for asymptotic optimality and widely used in communication systems. As the code length decreases, however, the performance of LT codes using RSD degrades. This is because the output ripple size becomes smaller and more likely to decrease to zero during early stage, leading to frequent premature decoding termination. As a result, RSD is unsuitable for multimedia communication, which typically involves transmission of small or segmented data, such as music files and Group of Pictures (GOP) in a coded video.


% since its expected output ripple size is relatively small during early decoding stage where decoding terminations 


%In the nonuniversal approach, we increase the degree-$1$ proportion in RSD. The output ripple size becomes larger accordingly, and the problem of premature decoding termination is then relieved. The proportion of low degrees, except for degree-$1$, is also decreased to reduce the number of redundant output symbols. In addition, we introduce \emph{Non-Repetitive} (NR) encoding scheme to avoid generating repeated degree-$1$ output symbols. NR encoding scheme limits the randomness of the encoding process so that its performance depends on the channel condition. Moreover, we integrate multiple NR encoders to achieve Unequal Erasure Protection (UEP) for Scalable Video Coding (SVC) layers with different importance. Experimental results show that our UEP scheme outperforms previous studies in terms of the Peak Signal-to-Noise Ratio (PSNR).



%In the universal approach, RSD is replaced by our proposed distributions. Meanwhile, we keep the encoding process intact to maintain the universality of LT codes. By minimizing objective functions under certain constraints, degree distributions can be constructed with their expected output ripple size approximating predetermined curves. Sequential Quadratic Programming (SQP) algorithm is employed to solve the minimization problem. Compared to RSD and previous works, our proposed ripple-based distribution (RBD) is able to reduce the average overhead needed to fully decode input symbols as well as the encoding and decoding complexity. Moreover, we record User Datagram Protocol (UDP) packet loss patterns over 802.11g WLAN and apply them to our simulations. The corresponding results show that the transmission efficiency can be improved by using RBD instead of RSD. Compared to RSD, RBD reduces the encoding and decoding complexity by at least $31.2\%$ and $25.4\%$, respectively.