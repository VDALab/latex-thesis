%!TEX root = Thesis.tex
%
% this file is encoded in utf-8
% v2.0 (Apr. 5, 2009)
% do not change the content of this file
% unless the thesis layout rule is changed
% 無須修改本檔內容,除非校方修改了
% 封面、書名頁、中文摘要、英文摘要、誌謝、目錄、表目錄、圖目錄、符號說明
% 等頁之格式


% default variables definitions
% 此處只是預設值,不需更改此處
% 請更改 my_names.tex 內容
\newcommand\cTitle{論文題目}
\newcommand\eTitle{MY THESIS TITLE}
\newcommand\myCname{我的中文名字}
\newcommand\myEname{My Name}
\newcommand\advisorCnameA{第一指導教授中文名字}
\newcommand\advisorEnameA{Name of the 1st Advisor}
\newcommand\advisorCnameB{第二指導教授中文名字}
\newcommand\advisorEnameB{Name of the 2nd Advisor}
\newcommand\advisorCnameC{第三指導教授中文名字}
\newcommand\advisorEnameC{Name of the 3rd Advisor}
\newcommand\univCname{大學名稱}
\newcommand\univEname{University}
\newcommand\deptCname{系所名稱}
\newcommand\fulldeptEname{Department}
\newcommand\deptEname{Department}
\newcommand\collEname{College}
\newcommand\degreeCname{學位}
\newcommand\degreeEname{Degree}
\newcommand\cYear{民國年份}
\newcommand\cMonth{月數}
\newcommand\eYear{Year}
\newcommand\eMonth{Month}
\newcommand\ePlace{Location of University}

% user's names; to replace those default variable definitions
% this file is encoded in utf-8
% v2.0 (Apr. 5, 2009)
% 填入你的論文題目、姓名等資料
% 如果題目內有必須以數學模式表示的符號,請用 \mbox{} 包住數學模式,如下範例
% 如果中文名字是單名,與姓氏之間建議以全形空白填入,如下範例
% 英文名字中的稱謂,如 Prof. 以及 Dr.,其句點之後請以不斷行空白~代替一般空白,如下範例
% 如果你的指導教授沒有如預設的三位這麼多,則請把相對應的多餘教授的中文、英文名
%    的定義以空的大括號表示
%    如,\renewcommand\advisorCnameB{}
%          \renewcommand\advisorEnameB{}
%          \renewcommand\advisorCnameC{}
%          \renewcommand\advisorEnameC{}

% 論文題目 (中文)
\renewcommand\cTitle{類比電路自動化與遷移設計之研究}

% 論文題目 (英文)
\renewcommand\eTitle{Research on Analog Design Automation and Migration}

% 我的姓名 (中文)
\renewcommand\myCname{潘柏丞}

% 我的姓名 (英文)
\renewcommand\myEname{Po-Cheng Pan}

% 指導教授A的姓名 (中文)
\renewcommand\advisorCnameA{陳宏明}

% 指導教授A的姓名 (英文)
\renewcommand\advisorEnameA{Hung-Ming Chen}

% 指導教授B的姓名 (中文)
\renewcommand\advisorCnameB{}

% 指導教授B的姓名 (英文)
\renewcommand\advisorEnameB{}

% 指導教授C的姓名 (中文)
\renewcommand\advisorCnameC{}

% 指導教授C的姓名 (英文)
\renewcommand\advisorEnameC{}

% 校名 (中文)
\renewcommand\univCname{國立交通大學}

% 校名 (英文)
\renewcommand\univEname{National Chiao Tung University}

% 系所名 (中文)
\renewcommand\deptCname{電子工程學系電子研究所}

% 系所全名 (英文)
\renewcommand\fulldeptEname{Department of Electronics Engineering \& Institute of Electronics}

% 系所短名 (英文, 用於書名頁學位名領域)
\renewcommand\deptEname{Electronics Engineering}

% 學院英文名 (如無,則以空的大括號表示)
\renewcommand\collEname{College of Electrical and Computer Engineering}

% 學位名 (中文)
\renewcommand\degreeCname{博士}

% 學位名 (英文)
\renewcommand\degreeEname{Doctor of Philosophy}

% 口試年份 (中文、民國)
\renewcommand\cYear{一零四}

% 口試月份 (中文)
\renewcommand\cMonth{四} 

% 口試年份 (阿拉伯數字、西元)
\renewcommand\eYear{2015} 

% 口試月份 (英文)
\renewcommand\eMonth{April}

% 學校所在地 (英文)
\renewcommand\ePlace{Hsinchu, Taiwan, Republic of China}

%畢業級別;用於書背列印;若無此需要可忽略
\newcommand\GraduationClass{102}

%%%%%%%%%%%%%%%%%%%%%%


% make the line spacing in effect
%\newcommand{\mybaselinestretch}{1.5}  %行距 1.5 倍 + 20%, (約為 double space)
%\renewcommand{\baselinestretch}{\mybaselinestretch}  % 論文行距預設值
%\parskip = 2ex                                       % 段落之間的間隔為兩個 x 的高度
%\parindent = 0Pt                                     % 這裡設成不內縮

%\renewcommand{\baselinestretch}{\mybaselinestretch}
\large % it needs a font size changing command to be effective

\newcommand\itsempty{}
%%%%%%%%%%%%%%%%%%%%%%%%%%%%%%%
%        cover 封面
%%%%%%%%%%%%%%%%%%%%%%%%%%%%%%%

\begin{titlepage}
% no page number
% next page will be page 1

\ifx\mywatermark\undefined 
  \thispagestyle{empty}  % 無頁碼、無 header (無浮水印)
\else
  \thispagestyle{EmptyWaterMarkPage} % 無頁碼、有浮水印
\fi

% aligned to the center of the page
\begin{center}
% font size (relative to 12 pt):
% \large (14pt) < \Large (18pt) < \LARGE (20pt) < \huge (24pt)< \Huge (24 pt)
%
%\makebox[12cm][s]{\Huge{\univCname}}\\  %顯示中文校名
\makebox[12cm][s]{\fontsize{36pt}{10pt}\selectfont{\univCname}}\\
\vspace{1.5cm}
%\makebox[12cm][s]{\huge{\deptCname}}\\ %顯示中文系所名
\makebox[12cm][s]{\fontsize{28pt}{10pt}\selectfont{\deptCname}}\\
\vspace{1.5cm}
%\makebox[6cm][s]{\huge{\degreeCname 論文}}\\ %顯示論文種類 (中文)
\makebox[6cm][s]{\fontsize{28pt}{10pt}\selectfont{\degreeCname 論文}}\\
\vspace{2cm}
%
% Set the line spacing to single for the titles (to compress the lines)
%\renewcommand{\baselinestretch}{1}   %行距 1 倍
\large % it needs a font size changing command to be effective
\LARGE{\cTitle}\\  % 中文題目
%
%\vspace{1cm}
%
\LARGE{\eTitle}\\ %英文題目
\vspace{2cm}
% \makebox is a text box with specified width;
% option s: stretch
% use \makebox to make sure
% 「研究生:」 與「指導教授:」occupy the same width
\hspace{4.5cm} \makebox[4cm][s]{\LARGE{研究生}} \makebox[0.5cm][c]{\LARGE{:}}
\LARGE{\myCname}  % 顯示作者中文名
\hfill \makebox[1cm][s]{}\\
%
\hspace{4.5cm} \makebox[4cm][s]{\LARGE{指導教授}}  \makebox[0.5cm][c]{\LARGE{:}}
\LARGE{\advisorCnameA}  %顯示指導教授A中文名
\hfill \makebox[1cm][s]{}\\
%
% 判斷是否有共同指導的教授 B
\ifx \advisorCnameB  \itsempty
\relax % 沒有 B 教授,所以不佔版面,不印任何空白
\else
% 共同指導的教授 B
\hspace{4.5cm} \makebox[4cm][s]{} \makebox[0.5cm][c]{}
\LARGE{\advisorCnameB}  %顯示指導教授B中文名
\hfill \makebox[1cm][s]{}\\
\fi
%
% 判斷是否有共同指導的教授 C
\ifx \advisorCnameC  \itsempty
\relax % 沒有 C 教授,所以不佔版面,不印任何空白
\else
% 共同指導的教授 C
\hspace{4.5cm} \makebox[3cm][s]{}
\LARGE{\advisorCnameC}  %顯示指導教授C中文名
\hfill \makebox[1cm][s]{}\\
\fi
%
\vfill
\makebox[10cm][s]{\LARGE{中華民國\cYear 年\cMonth 月}}%
%
\end{center}
% Resume the line spacing to the desired setting
%\renewcommand{\baselinestretch}{\mybaselinestretch}   %恢復原設定
% it needs a font size changing command to be effective
% restore the font size to normal
\normalsize
\end{titlepage}
%%%%%%%%%%%%%%

%% 從摘要到本文之前的部份以小寫羅馬數字印頁碼
% 但是從「書名頁」(但不印頁碼) 就開始計算
\setcounter{page}{1}
\pagenumbering{roman}
%%%%%%%%%%%%%%%%%%%%%%%%%%%%%%%
%       書名頁 
%%%%%%%%%%%%%%%%%%%%%%%%%%%%%%%
\newpage

% 判斷是否要浮水印?
\ifx\mywatermark\undefined 
  \thispagestyle{empty}  % 無頁碼、無 header (無浮水印)
\else
  \thispagestyle{EmptyWaterMarkPage} % 無頁碼、有浮水印
\fi

%no page number
% create an entry in table of contents for 書名頁
%%\phantomsection % for hyperref to register this
%%\addcontentsline{toc}{chapter}{\nameInnerCover}


% aligned to the center of the page
\begin{center}
% font size (relative to 12 pt):
% \large (14pt) < \Large (18pt) < \LARGE (20pt) < \huge (24pt)< \Huge (24 pt)
% Set the line spacing to single for the titles (to compress the lines)
%%\renewcommand{\baselinestretch}{1}   %行距 1 倍
% it needs a font size changing command to be effective
%中文題目
\Large{\cTitle}\\ %%%%%
%\vspace{1cm}
% 英文題目
\Large{\eTitle}\\ %%%%%
%\vspace{1cm}
\vfill
% \makebox is a text box with specified width;
% option s: stretch
% use \makebox to make sure
% 「研究生:」 與「指導教授:」occupy the same width
\makebox[3cm][s]{\large{研 究 生}} \makebox{\large{:}}
\makebox[3cm][l]{\large{\myCname}} %%%%%
\hfill
\makebox[1.5cm][s]{\large{Student}} \makebox{\large{:}}
\makebox[5cm][l]{\large{\myEname}}\\ %%%%%
%
%\vspace{1cm}
%
\makebox[3cm][s]{\large{指導教授}} \makebox{\large{:}}
\makebox[4cm][l]{\large{\advisorCnameA} \large{博士}} %%%%%
%\makebox[1cm][l]{\large{博士}}
\hfill
\makebox[1.5cm][s]{\large{Advisor}} \makebox{\large{:}}
\makebox[5cm][l]{\large{Dr. \advisorEnameA}}\\ %%%%%
%
% 判斷是否有共同指導的教授 B
\ifx \advisorCnameB  \itsempty
\relax % 沒有 B 教授,所以不佔版面,不印任何空白
\else
%共同指導的教授B
\makebox[3cm][s]{}
%\makebox[3cm][l]{\large{\advisorCnameB}} %%%%%
\makebox[0.5cm][l]{}
\makebox[4cm][l]{\large{\advisorCnameB} \large{博士}} %%%%%
\hfill
\makebox[2cm][s]{}
\makebox[5cm][l]{\large{Dr. \advisorEnameB}}\\ %%%%%%%%%%%%%%%%%%%%%%%%%%%%%%%%%%%%%%%%%%%%%%%%%%%%%%%%%%%%%%%%%%%%%%%%%%%%%%%%%%%%%%%%%%%%%%%%%%%%%%%%%%%%%%%%%%%%%
\fi
%
% 判斷是否有共同指導的教授 C
\ifx \advisorCnameC  \itsempty
\relax % 沒有 C 教授,所以不佔版面,不印任何空白
\else
%共同指導的教授C
\makebox[3cm][s]{}
\makebox[3cm][l]{\large{\advisorCnameC}} %%%%%
\hfill
\makebox[2cm][s]{}
\makebox[5cm][l]{\large{\advisorEnameC}}\\ %%%%%
\fi
%
% Resume the line spacing to the desired setting
%\renewcommand{\baselinestretch}{\mybaselinestretch}   %恢復原設定
\large %it needs a font size changing command to be effective
%
\vfill
\makebox[4cm][s]{\large{\univCname}}\\% 校名
\makebox[6cm][s]{\large{\deptCname}}\\% 系所名
\makebox[3cm][s]{\large{\degreeCname 論文}}\\% 學位名
%
%\vspace{1cm}
\vfill
%%\large{A Dissertation}\\%
\normalsize{A Dissertation}\\
%%\large{Submitted to}%
\normalsize{Submitted to}
%
%%\large{\fulldeptEname}\\%系所全名 (英文)
\normalsize{\fulldeptEname}\\
%
%
\ifx \collEname  \itsempty
\relax % 沒有學院名 (英文),所以不佔版面,不印任何空白
\else
% 有學院名 (英文)
%%\large{\collEname}\\% 學院名 (英文)
\normalsize{\collEname}\\
\fi
%
%%\large{\univEname}\\%校名 (英文)
\normalsize{\univEname}\\
%
%%\large{in Partial Fulfillment of the Requirements}\\
\normalsize{in Partial Fulfillment of the Requirements}\\
%
%%\large{for the Degree of}
\normalsize{for the Degree of}
%
%%\large{\degreeEname}\\%學位名(英文)
\normalsize{\degreeEname}\\
%
%%\large{in}\\
\normalsize{in}\\
%
%%\large{\deptEname}\\%系所短名(英文;表明學位領域)
\normalsize{\deptEname}\\
%
%%\large{\eMonth\ \eYear}\\%月、年 (英文)
\normalsize{\eMonth\ \eYear}\\
%
%%\large{\ePlace}% 學校所在地 (英文)
\normalsize{\ePlace}% 學校所在地 (英文)
\vfill
\large{中華民國}%
\large{\cYear}% %%%%%
\large{年}%
\large{\cMonth}% %%%%%
\large{月}\\
\end{center}
% restore the font size to normal
\normalsize
\clearpage


%%%%%%%%%%%%%%%%%%%%%%%%%%%%%%%
%       論文口試委員審定書 (計頁碼,但不印頁碼) 
%%%%%%%%%%%%%%%%%%%%%%%%%%%%%%%


%%%%%%%%%%%%%%%%%%%%%%%%%%%%%%%
%       授權書 (計頁碼,但不印頁碼) 
%%%%%%%%%%%%%%%%%%%%%%%%%%%%%%%

























%%%%%%%%%%%%%%%%%%%%%%%%%%%%%%%%
%%       中文摘要 
%%%%%%%%%%%%%%%%%%%%%%%%%%%%%%%%
%
\newpage
\thispagestyle{plain}  % 無 header,但在浮水印模式下會有浮水印

% create an entry in table of contents for 中文摘要
%%\phantomsection % for hyperref to register this
%%\addcontentsline{toc}{chapter}{\nameCabstract}

% aligned to the center of the page
\begin{center}
% font size (relative to 12 pt):
% \large (14pt) < \Large (18pt) < \LARGE (20pt) < \huge (24pt)< \Huge (24 pt)
% Set the line spacing to single for the names (to compress the lines)
%%\renewcommand{\baselinestretch}{1}   %行距 1 倍
% it needs a font size changing command to be effective
\Large{\cTitle}\\  %中文題目
\vspace{0.83cm}
% \makebox is a text box with specified width;
% option s: stretch
% use \makebox to make sure
% each text field occupies the same width
\makebox[1.5cm][s]{\large{學生:}}
\makebox[3cm][l]{\large{\myCname}} %學生中文姓名
\hfill
%
\makebox[3cm][s]{\large{指導教授:}}
\makebox[3cm][l]{\large{\advisorCnameA}\large{教授}} \\ %教授A中文姓名
%
% 判斷是否有共同指導的教授 B
\ifx \advisorCnameB  \itsempty
\relax % 沒有 B 教授,所以不佔版面,不印任何空白
\else
%共同指導的教授B
\makebox[1.5cm][s]{}
\makebox[3cm][l]{} %%%%%
\hfill
\makebox[3cm][s]{}
\makebox[3cm][l]{\large{\advisorCnameB}\large{教授}}\\ %教授B中文姓名
\fi
%
% 判斷是否有共同指導的教授 C
\ifx \advisorCnameC  \itsempty
\relax % 沒有 C 教授,所以不佔版面,不印任何空白
\else
%共同指導的教授C
\makebox[1.5cm][s]{}
\makebox[3cm][l]{} %%%%%
\hfill
\makebox[3cm][s]{}
\makebox[3cm][l]{\large{\advisorCnameC}\large{教授}}\\ %教授C中文姓名
\fi
%
\vspace{0.42cm}
%
\large{\univCname}\large{\deptCname}\\ %校名系所名
\vspace{0.83cm}
%\vfill
\makebox[2.5cm][s]{\large{摘要}}\\
\end{center}
% Resume the line spacing to the desired setting
%%\renewcommand{\baselinestretch}{\mybaselinestretch}   %恢復原設定
%it needs a font size changing command to be effective
% restore the font size to normal
\normalsize
%%%%%%%%%%%%%
% !TEX root = Thesis.tex

在積體電路的範疇,為了更有效地達到先進製程底下的電路效能,類比電路的自動化已不容忽視。再更深入地探討類比電路合成以及佈局設計自動化,有條件性地電路設計將變得更為重要。當我們探討類比電路合成時,元件模型、效能需求以及寄生電容效應構成了探索電路方程式的限制條件。然而,清楚掌握電路元件的非線性方程式也同樣十分複雜與耗時。除此之外,由於電路的效能對於實體佈局設計也相當敏感,佈局相依效應在佈局設計時也需要一併考慮。傳統上,佈局設計為了能夠滿足生產的電路效能需求,大多仰賴曠日費時的人工調整。更甚者,隨著先進製程的演進,元件模型的複雜度以及佈局設計的限制條件也不斷提升。倘若與設計相依的限制條件能在自動化合成與佈局設計的同時被重視,先進製程底下類比電路的困難度也會隨之減緩。

在這篇論文中,我們提出了一套從電路合成到佈局設計產生的類比電路遷移設計演算法。該套件可被拆分為三個階段,基於平行化基因演算法的電路效能探索、基於整合性限制條件的類比電路繞線以及類比電路敏捷佈局遷移雛形開發。為了滿足遷移設計所要實現的內容,目標製程的電路合成以及實體佈局產生都需要被一一實踐。首先,基於平行化基因演算法的效能探索可以分析出特定積體電路製程底下的設計極限。我們所提出的方法不只是逐一檢視在各個設計邊界的特性,基因演算法勾勒出該製程的效能區間供最佳化設計。實驗結果展現了我們的基因演算法以及機率分布式的電路模擬。我們所提出的效能探索可以有效且迅速地在射頻分散式放大器以及運算放大器基於不同製程下找到效能極限並設計之。

接著我們提出了一個整合式限制條件的類比電路繞線方法,該方法提供了一個整合電路的設計限制以及製程限制的概念以實現業界等級的電路設計產品。事實上,大部份的設計限制條件經由積體電路製造商所提供的僅限於在實體佈局上的限制條件,該條件並未實際考慮由前端電路設計提供之設計條件。然而,這些設計限制相當地重要。為了實現實體佈局電路設計自動化,我們提出了整合性設計限制條件組合。同時我們也將這個概念實際運用在tsmc的40奈米製程電路上。該電路由於保存了前端設計的限制條件,線路的對稱性提升了電路的訊號穩定性。我們同時也測試了不同的限制條件對於繞線的順序造成了電路效能的顯著影響。

最後,我們在論文中提出了一種敏捷的類比電路設計遷移方法。在先進製程中,由於設計限制條件倍增以及效能需求倍增,類比電路自動化逐漸變得更複雜。不僅如此,寄生電路效應嚴重以及製造可靠性不確定讓情況變得更加複雜。為了有效參考現有的電路設計,我們提出的電路遷移方法能夠萃取不只是佈局擺放特性,也保存了繞線與佈局擺放之間的關聯性。讓遷移設計時的繞線擺放更為簡單與有依據。除此之外,由於佈局擺放的彈性增加,我們的遷移方法可以產生多組的雛形設計以供選則。不僅如此,該遷移方法也能針對設計中的每一條繞線調整最佳化。實驗結果展現了產生多組佈局設計結果的可行性,在不影響整體電路效能的前提下,也有效提升了自動化繞線的完成度。

總而言之,我們的類比電路遷移設計演算法,實現了在先進製程底下從電路合成到佈局雛形產生類比電路並符合需求,同時我們的實驗結果也展示了我們所提出方法的正確性與效率。

























%%%%%%%%%%%%%%%%%%%%%%%%%%%%%%%
%       英文摘要 
%%%%%%%%%%%%%%%%%%%%%%%%%%%%%%%
%
\newpage
\thispagestyle{plain}  % 無 header,但在浮水印模式下會有浮水印

% create an entry in table of contents for 英文摘要
%%\phantomsection % for hyperref to register this
%%\addcontentsline{toc}{chapter}{\nameEabstract}

% aligned to the center of the page
\begin{center}
% font size:
% \large (14pt) < \Large (18pt) < \LARGE (20pt) < \huge (24pt)< \Huge (24 pt)
% Set the line spacing to single for the names (to compress the lines)
%%\renewcommand{\baselinestretch}{1}   %行距 1 倍
%\large % it needs a font size changing command to be effective
\Large{\eTitle}\\  %英文題目
\vspace{0.83cm}
% \makebox is a text box with specified width;
% option s: stretch
% use \makebox to make sure
% each text field occupies the same width
\makebox[2cm][s]{\large{Student: }}
\makebox[5cm][l]{\large{\myEname}} %學生英文姓名
\hfill
%
\makebox[2cm][s]{\large{Advisor: }}
\makebox[5cm][l]{\large{\advisorEnameA}} \\ %教授A英文姓名
%
% 判斷是否有共同指導的教授 B
\ifx \advisorCnameB  \itsempty
\relax % 沒有 B 教授,所以不佔版面,不印任何空白
\else
%共同指導的教授B
\makebox[2cm][s]{}
\makebox[5cm][l]{} %%%%%
\hfill
\makebox[2cm][s]{}
\makebox[5cm][l]{\large{\advisorEnameB}}\\ %教授B英文姓名
\fi
%
% 判斷是否有共同指導的教授 C
\ifx \advisorCnameC  \itsempty
\relax % 沒有 C 教授,所以不佔版面,不印任何空白
\else
%共同指導的教授C
\makebox[2cm][s]{}
\makebox[5cm][l]{} %%%%%
\hfill
\makebox[2cm][s]{}
\makebox[5cm][l]{\large{\advisorEnameC}}\\ %教授C英文姓名
\fi
%
\vspace{0.42cm}
%%\large{Submitted to }\large{\fulldeptEname}\\  %英文系所全名
\large{\fulldeptEname}\\  %英文系所全名
%
%  建議不用加學院名稱
\ifx \collEname  \itsempty
\relax % 如果沒有學院名 (英文),則不佔版面,不印任何空白
\else
% 有學院名 (英文)
\large{\collEname}\\% 學院名 (英文)
\fi
%
\large{\univEname}\\  %英文校名
\vspace{0.83cm}
%\vfill
%
\large{ABSTRACT}\\
%\vspace{0.5cm}
\end{center}
% Resume the line spacing the desired setting
%%\renewcommand{\baselinestretch}{\mybaselinestretch}   %恢復原設定
%\large %it needs a font size changing command to be effective
% restore the font size to normal
\normalsize
%%%%%%%%%%%%%
% !TEX root = Thesis.tex

This dissertation presents an analog automation flow and design migration methodology
%both nonuniversal and universal approaches that improve the performance of short-length Luby Transform (LT) codes with applications to multimedia communication. The key idea is to increase the expected \emph{output ripple size} to prevent Belief-Propagation (BP) decoding process from terminating prematurely. The nonuniversal approach with Soliton-based distribution and randomness-limited encoding scheme is discussed first. Then the universal approach is investigated focusing on constructing degree distributions with respect to the output ripple size.


%, where the output ripple size is the number of degree-$1$ output symbols


%LT codes have been proved to be capacity-achieving on erasure channels. The performance of these codes is dominated by the code length and the design of the degree distribution. The state of the art robust Soliton distribution (RSD) is designed for asymptotic optimality and widely used in communication systems. As the code length decreases, however, the performance of LT codes using RSD degrades. This is because the output ripple size becomes smaller and more likely to decrease to zero during early stage, leading to frequent premature decoding termination. As a result, RSD is unsuitable for multimedia communication, which typically involves transmission of small or segmented data, such as music files and Group of Pictures (GOP) in a coded video.


% since its expected output ripple size is relatively small during early decoding stage where decoding terminations 


%In the nonuniversal approach, we increase the degree-$1$ proportion in RSD. The output ripple size becomes larger accordingly, and the problem of premature decoding termination is then relieved. The proportion of low degrees, except for degree-$1$, is also decreased to reduce the number of redundant output symbols. In addition, we introduce \emph{Non-Repetitive} (NR) encoding scheme to avoid generating repeated degree-$1$ output symbols. NR encoding scheme limits the randomness of the encoding process so that its performance depends on the channel condition. Moreover, we integrate multiple NR encoders to achieve Unequal Erasure Protection (UEP) for Scalable Video Coding (SVC) layers with different importance. Experimental results show that our UEP scheme outperforms previous studies in terms of the Peak Signal-to-Noise Ratio (PSNR).



%In the universal approach, RSD is replaced by our proposed distributions. Meanwhile, we keep the encoding process intact to maintain the universality of LT codes. By minimizing objective functions under certain constraints, degree distributions can be constructed with their expected output ripple size approximating predetermined curves. Sequential Quadratic Programming (SQP) algorithm is employed to solve the minimization problem. Compared to RSD and previous works, our proposed ripple-based distribution (RBD) is able to reduce the average overhead needed to fully decode input symbols as well as the encoding and decoding complexity. Moreover, we record User Datagram Protocol (UDP) packet loss patterns over 802.11g WLAN and apply them to our simulations. The corresponding results show that the transmission efficiency can be improved by using RBD instead of RSD. Compared to RSD, RBD reduces the encoding and decoding complexity by at least $31.2\%$ and $25.4\%$, respectively.

%%%%%%%%%%%%%%%%%%%%%%%%%%%%%%%
%       誌謝 
%%%%%%%%%%%%%%%%%%%%%%%%%%%%%%%
%
% Acknowledgement
\newpage
\thispagestyle{plain}
%\chapter*{\protect\makebox[5cm][s]{\nameAckn}} %\makebox{} is fragile; need protect
%\phantomsection % for hyperref to register this
%\addcontentsline{toc}{chapter}{\nameAckn}
\begin{center}
\makebox[5cm][s]{\Large{\textbf{誌謝}}}
\end{center}
\vspace{2cm}
在漫長的五年博士班的修業過程中,如今下筆要寫出這段誌謝誠然百感交集
%承蒙師長的教誨、家人的支持以及夥伴們的合作,最終我才能完成學業。
%特別感謝張錫嘉老師的指導與提供的資源,並給予我嘗試錯誤的機會,能夠在這樣自由的環境中從事研究,我感到十分的幸福。
%同時感謝邵家健教授,讓我在修業過程中學習到寶貴的研究方法和態度。
%此外,由衷感激廖彥欽學姊、陳志龍學長及翁政吉學長,因為有你們的幫助,才能加快我研究的腳步。
%也感謝OASIS實驗室及OCEAN和邵家健教授團隊的所有成員,透過日常的討論,解答我許多課業上的疑問。
%最後要感謝我家人全心全意的支持,我才能堅定的達成每個目標。





%%%%%%%%%%%%%%%%%%%%%%%%%%%%%%%
%       目錄 
%%%%%%%%%%%%%%%%%%%%%%%%%%%%%%%
%
% Table of contents
%\newpage
\renewcommand{\contentsname}{Contents}                                                                 
%\renewcommand{\contentsname}{\protect\makebox[5cm][s]{\nameToc}}
%%\makebox{} is fragile; need protect
%\phantomsection % for hyperref to register this
%\addcontentsline{toc}{chapter}{\nameToc}
%\addcontentsline{toc}{chapter}{\contentsname}
\tableofcontents

%%%%%%%%%%%%%%%%%%%%%%%%%%%%%%%
%       表目錄 
%%%%%%%%%%%%%%%%%%%%%%%%%%%%%%%
%
% List of Tables
%\newpage
%\renewcommand{\listtablename}{\protect\makebox[5cm][s]{\nameLot}}
%%\makebox{} is fragile; need protect
%\phantomsection % for hyperref to register this
%\addcontentsline{toc}{chapter}{\nameLot}
%\addcontentsline{toc}{chapter}{\listtablename}
\listoftables

%%%%%%%%%%%%%%%%%%%%%%%%%%%%%%%
%       圖目錄 
%%%%%%%%%%%%%%%%%%%%%%%%%%%%%%%
%
% List of Figures
%\newpage
%\renewcommand{\listfigurename}{\protect\makebox[5cm][s]{\nameTof}}
%%\makebox{} is fragile; need protect
%\phantomsection % for hyperref to register this
%\addcontentsline{toc}{chapter}{\nameTof}
%\addcontentsline{toc}{chapter}{\listfigurename}
\listoffigures

%%%%%%%%%%%%%%%%%%%%%%%%%%%%%%%
%       符號說明 
%%%%%%%%%%%%%%%%%%%%%%%%%%%%%%%
%
% Symbol list
% define new environment, based on standard description environment
% adapted from p.60~64, <<The LaTeX Companion>>, 1994, ISBN 0-201-54199-8
\newcommand{\SymEntryLabel}[1]%
  {\makebox[3cm][l]{#1}}
%
\newenvironment{SymEntry}
   {\begin{list}{}%
       {\renewcommand{\makelabel}{\SymEntryLabel}%
        \setlength{\labelwidth}{3cm}%
        \setlength{\leftmargin}{\labelwidth}%
        }%
   }%
   {\end{list}}
%%
\newpage
\chapter*{\protect\makebox[5cm][s]{\nameSlist}} %\makebox{} is fragile; need protect
%\addcontentsline{toc}{chapter}{\nameSlist}
% !TEX root = Thesis.tex
%\phantomsection
\begin{SymEntry}

% symbol for chapter1
\item[LDE]
Layout dependent effect 

\item[RFDA]
Radio-frequency distributed amplifier

\item[DRC]
Design rule check

\item[LVS]
Layout versus schematic

\item[MOOP]
Multi-objective optimization problem

\item[PDK]
Process design kit

\item[DFM]
Design for manufacturability

%symbols for chapter2

\item[PGA]
Parallel genetic algorithm

\item[$V^D$]
Device-level variable set

\item[$V^C$]
Circuit-level variable set

\item[$S_{V^D}$]
Step number of $V^D$

\item[$V^P$]
Parasitic variable set

\item[$Z_{K\times S}$]
A set of performance space matrix

\item[K]
Number of performance specification types

\item[G]
A random set of chromosome, for each $g_i \in G$ randomly obtains value of the $k^{th}$ specification from $z_{k,1}$ to $z_{k,S}$ 

\item[P]
Major population generation according to $Z_{K\times S}$

\item[R]
The corresponding result set from G

\item[$M_P$]
Migration rate for PGA

\item[$\Psi $]
A set of overall device-level variable

\item[$N_{V^D}$]
The number of all device variables type

%symbols for chapter3
\item[P\&R]
Placement and routing

\item[HCDR]
Hierarchical constraint driven routing

\item[$T$]
Top-level Design with cell blocks

\item[$T'$]
Design T after partition  

\item[$T^r$]
T with partial routing completed

\item[$CG_T$]
Constraint group of the current level design

\item[$N$]
List of connectivity w.r.t. the design T

\item[$Con  $]
Constraint pop out from $CG_T$

\item[$SymCon$]
Symmetrical constraint in $CG_T$

\item[$MatchCon$]
Matching constraint in $CG_T$

\item[$getCG$]
Obtaining constraint group for partitioned design

\item[$SymGrp$]
Symmetrical group described in SymCon

\item[$MatchGrp$]
Matching group described in MatchCon

\item[$num(CG_T)$]
Total number of constraints in $CG_T$

\item[$S$]
Topology slicing tree for placement

\item[$G_{C_r}$]
Crossing graph for routing 

\item[CDT]
Constrained Delaunay triangulation

\item[VGA]
Variable-gained amplifier

\item[LDO]
Low drop-out regulator

\item[RtMaMi]
Routing manually migrated into targeting technology

\item[RtNoMi]
Routing is not migrated into targeting technology.

\item[WL]
Wirelength

\item[WS]
Wire segment


%\item[FEC]
%Forward Error/Erasure Correction
%
%\item[GOP]
%Group of Pictures
%
%\item[IDR]
%Instantaneous Decoder Refresh
%\item[ISD]
%Ideal Soliton Distribution
%
%\item[KKT]
%Karush-Kuhn-Tucker
%
%\item[LDPC]
%Low-Density Parity-Check
%\item[LIB]
%Least Important Bits
%\item[LT]
%Luby Transform
%
%
%\item[MDS]
%Maximum Distance Separable
%\item[MIB]
%Most Important Bits
%\item[MRSD]
%Modified Robust Soliton Distribution
%
%\item[NALU]
%Network Abstraction Layer Units
%\item[NR]
%Non-Repetitive
%
%
%\item[PSNR]
%Peak Signal-to-Noise Ratio
%
%\item[RBD]
%Ripple-Based Distribution
%\item[RS]
%Reed-Solomon
%\item[RSD]
%Robust Soliton Distribution
% 
%
%
%\item[SQP]
%Sequential Quadratic Programming
%\item[SVC]
%Scalable Video Coding
%\item[SW]
%Sliding-Window
%
%
%\item[UDP]
%User Datagram Protocol
%\item[UEP]
%Unequal Erasure Protection
%
%
%
%
%\item[$k$]
%Code length
%
%\item[$K$]
%Number of received output symbols
%
%\item[$\epsilon$]
%Overhead ($\epsilon=\frac{K-k}{k}$)
%
%\item[$\rho$]
%Number of decoded input symbols during BP decoding process
%
%\item[$\rho/k$]
%Ratio of decoded input symbols during BP decoding process
%
%\item[$\Omega^r(x)$]
%Robust Soliton distribution (RSD)
%
%\item[$\Omega^P(x)$]
%Ripple-Based Distribution (RBD)
%
%
%\item[$\Gamma(\rho)$]
%Occurrence probability of BP decoding termination when $\rho$ input %symbols are decoded
%
%\item[$\Pi(\rho)$]
%Cumulative probability of decoding termination ($\Pi(\rho)=\sum_{q=0}^{\%rho}{\Gamma(q)}$)
%
%\item[$P_e$]
%Channel erasure rate
%
%\item[$\overline{\Delta}$]
%Average number of symbol operations to generate an encoding symbol. (A %symbol operation is either an exclusive-or between two symbols or a copy %of one symbol to another.)
%
%\item[$\overline{s}$]
%Average number of symbol operations needed to decode an input symbol
%
%\item[$\overline{m}$]
%Average number of output symbols transmitted for decoding $k$ input %symbols
%
%\item[$d(c_l)$]
%Degree of output symbol $c_l$
%
%\item[$C_r$]
%A set of $K$ received output symbols
%
%\item[$S$]
%Successful decoding probability (Probability of decoding all $k$ input %symbols from $(1+\epsilon)k$ received encoding symbols, where $S=\Gamma(%k)=1-\sum_{q=0}^{k-1}{\Gamma(q)}=1-\Pi(k-1)$)
%
%\item[$L$]
%Symbol loss probability. (Probability that an input symbol remains undecoded after BP decoding process, where $L=\sum_{\rho=0}^{k}{\frac{k-\rho}{k}\Gamma(\rho)}$.)



%$\Lambda(c_l)$ &: A set of input symbols connected by encoding symbol $c_l$ (neighbors of encoding symbol $c_l$). \\
%$\Lambda(i_l)$ &: A set of encoding symbols connecting to input symbol $i_l$ (neighbors of input symbol $i_l$). \\
%$R(\rho)$ &: Average number of degree-$1$ encoding symbols during BP decoding process when $\rho$ input symbols are decoded. \\


\end{SymEntry}


\newpage
%%% 論文本體頁碼回復為阿拉伯數字計頁,並從頭起算
\pagenumbering{arabic}
%%%%%%%%%%%%%%%%%%%%%%%%%%%%%%%%