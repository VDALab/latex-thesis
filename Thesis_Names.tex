% this file is encoded in utf-8
% v2.0 (Apr. 5, 2009)
% 填入你的論文題目、姓名等資料
% 如果題目內有必須以數學模式表示的符號,請用 \mbox{} 包住數學模式,如下範例
% 如果中文名字是單名,與姓氏之間建議以全形空白填入,如下範例
% 英文名字中的稱謂,如 Prof. 以及 Dr.,其句點之後請以不斷行空白~代替一般空白,如下範例
% 如果你的指導教授沒有如預設的三位這麼多,則請把相對應的多餘教授的中文、英文名
%    的定義以空的大括號表示
%    如,\renewcommand\advisorCnameB{}
%          \renewcommand\advisorEnameB{}
%          \renewcommand\advisorCnameC{}
%          \renewcommand\advisorEnameC{}

% 論文題目 (中文)
\renewcommand\cTitle{敏捷類比電路合成與布局設計遷移方法}

% 論文題目 (英文)
\renewcommand\eTitle{Methodologies on Rapid Analog Circuit Synthesis and layout migration}

% 我的姓名 (中文)
\renewcommand\myCname{潘柏丞}

% 我的姓名 (英文)
\renewcommand\myEname{Po-Cheng Pan}

% 指導教授A的姓名 (中文)
\renewcommand\advisorCnameA{陳宏明}

% 指導教授A的姓名 (英文)
\renewcommand\advisorEnameA{Hung-Ming Chen}

% 指導教授B的姓名 (中文)
\renewcommand\advisorCnameB{}

% 指導教授B的姓名 (英文)
\renewcommand\advisorEnameB{}

% 指導教授C的姓名 (中文)
\renewcommand\advisorCnameC{}

% 指導教授C的姓名 (英文)
\renewcommand\advisorEnameC{}

% 校名 (中文)
\renewcommand\univCname{國立交通大學}

% 校名 (英文)
\renewcommand\univEname{National Chiao Tung University}

% 系所名 (中文)
\renewcommand\deptCname{電子工程學系電子研究所}

% 系所全名 (英文)
\renewcommand\fulldeptEname{Department of Electronics Engineering \& Institute of Electronics}

% 系所短名 (英文, 用於書名頁學位名領域)
\renewcommand\deptEname{Electronics Engineering}

% 學院英文名 (如無,則以空的大括號表示)
\renewcommand\collEname{College of Electrical and Computer Engineering}

% 學位名 (中文)
\renewcommand\degreeCname{博士}

% 學位名 (英文)
\renewcommand\degreeEname{Doctor of Philosophy}

% 口試年份 (中文、民國)
\renewcommand\cYear{一零四}

% 口試月份 (中文)
\renewcommand\cMonth{五} 

% 口試年份 (阿拉伯數字、西元)
\renewcommand\eYear{2015} 

% 口試月份 (英文)
\renewcommand\eMonth{May}

% 學校所在地 (英文)
\renewcommand\ePlace{Hsinchu, Taiwan, Republic of China}

%畢業級別;用於書背列印;若無此需要可忽略
\newcommand\GraduationClass{103}

%%%%%%%%%%%%%%%%%%%%%%