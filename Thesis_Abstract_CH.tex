% !TEX root = Thesis.tex

在積體電路的範疇,為了更有效地達到先進製程底下的電路效能,類比電路的自動化已不容忽視。再更深入地探討類比電路合成以及佈局設計自動化,有條件性地電路設計將變得更為重要。當我們探討類比電路合成時,元件模型、效能需求以及寄生電容效應構成了探索電路方程式的限制條件。然而,清楚掌握電路元件的非線性方程式也同樣十分複雜與耗時。除此之外,由於電路的效能對於實體佈局設計也相當敏感,佈局相依效應在佈局設計時也需要一併考慮。傳統上,佈局設計為了能夠滿足生產的電路效能需求,大多仰賴曠日費時的人工調整。更甚者,隨著先進製程的演進,元件模型的複雜度以及佈局設計的限制條件也不斷提升。倘若與設計相依的限制條件能在自動化合成與佈局設計的同時被重視,先進製程底下類比電路的困難度也會隨之減緩。

在這篇論文中,我們提出了一套從電路合成到佈局設計產生的類比電路遷移設計演算法。該套件可被拆分為三個階段,基於平行化基因演算法的電路效能探索、基於整合性限制條件的類比電路繞線以及類比電路敏捷佈局遷移雛形開發。為了滿足遷移設計所要實現的內容,目標製程的電路合成以及實體佈局產生都需要被一一實踐。首先,基於平行化基因演算法的效能探索可以分析出特定積體電路製程底下的設計極限。我們所提出的方法不只是逐一檢視在各個設計邊界的特性,基因演算法勾勒出該製程的效能區間供最佳化設計。實驗結果展現了我們的基因演算法以及機率分布式的電路模擬。我們所提出的效能探索可以有效且迅速地在射頻分散式放大器以及運算放大器基於不同製程下找到效能極限並設計之。

接著我們提出了一個整合式限制條件的類比電路繞線方法,該方法提供了一個整合電路的設計限制以及製程限制的概念以實現業界等級的電路設計產品。事實上,大部份的設計限制條件經由積體電路製造商所提供的僅限於在實體佈局上的限制條件,該條件並未實際考慮由前端電路設計提供之設計條件。然而,這些設計限制相當地重要。為了實現實體佈局電路設計自動化,我們提出了整合性設計限制條件組合。同時我們也將這個概念實際運用在tsmc的40奈米製程電路上。該電路由於保存了前端設計的限制條件,線路的對稱性提升了電路的訊號穩定性。我們同時也測試了不同的限制條件對於繞線的順序造成了電路效能的顯著影響。

最後,我們在論文中提出了一種敏捷的類比電路設計遷移方法。在先進製程中,由於設計限制條件倍增以及效能需求倍增,類比電路自動化逐漸變得更複雜。不僅如此,寄生電路效應嚴重以及製造可靠性不確定讓情況變得更加複雜。為了有效參考現有的電路設計,我們提出的電路遷移方法能夠萃取不只是佈局擺放特性,也保存了繞線與佈局擺放之間的關聯性。讓遷移設計時的繞線擺放更為簡單與有依據。除此之外,由於佈局擺放的彈性增加,我們的遷移方法可以產生多組的雛形設計以供選則。不僅如此,該遷移方法也能針對設計中的每一條繞線調整最佳化。實驗結果展現了產生多組佈局設計結果的可行性,在不影響整體電路效能的前提下,也有效提升了自動化繞線的完成度。

總而言之,我們的類比電路遷移設計演算法,實現了在先進製程底下從電路合成到佈局雛形產生類比電路並符合需求,同時我們的實驗結果也展示了我們所提出方法的正確性與效率。