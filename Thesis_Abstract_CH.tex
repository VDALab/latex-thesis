本論文提出一種類比電路自動化以及遷移設計法則
%非普適性(Nonuniveral)及普適性(Universal)兩類方法,改善短長度LT碼的表現,並應用於多媒體通訊。
%基本概念在於增加單一連線字符(Degree One Symbol)的平均數量,避免置信傳播(Belief-Propagation,BP)解碼過程過早中斷。
%我們首先介紹非普適性方法,包含奠基於索靈頓的連線數機率分佈以及有限隨機性編碼法。
%接著分析普適性方法,並將重點放在,如何根據解碼過程中,單一連線字符的平均數量來建立連線數機率分佈。

%LT碼已經被證明,在抹除通道(Erasure Channel)上可達到通道容量,這一類型的通道編碼表現,主要取決於碼長以及連線數機率分佈的設計。
%目前最廣泛使用的健全索靈頓分佈(Robust Soliton Distribution,RSD),是針對無限碼長作最佳化設計,並廣泛使用於通訊系統中。
%然而當碼長逐漸降低時,LT碼結合健全索靈頓分佈的表現也隨之變差,因為單一連線字符的數量,變得更容易在解碼初期降到零,導致解碼過程經常過早中止。
%因此,健全索靈頓分佈不適合應用在多媒體通訊中,傳輸資料量小或分割後的檔案,例如音樂檔案、圖像群組(Group of Pictures,GOP)、等等。


%在本論文所提出的Nonuniveral方法中,我們提高單一連線字符在健全索靈頓分佈中的比例,避免解碼過程過早中止。
%而低連線字符的比例也被降低,使接收端收到冗餘字符的機率減少。
%我們還提出非重複性(Non-Repetitive,NR)編碼方法,旨在減少重複的單一連線字符的產生。
%但是非重複性編碼方法限制了編碼的隨機性,LT碼的表現因此跟通道狀況產生關聯性,所以此編碼方法為非普適性。
%此外,我們結合數個非重複性編碼器,來達到非均等抹除保護(Unequal Erasure Protection),給予可適性視訊編碼(Scalable Video Coding)中,不同重要性的層級相對應的保護。
%實驗結果顯示,跟過去的研究比較,我們所提出的非均等抹除保護方法,在峰值信噪比(Peak Signal-to-Noise Ratio,PSNR)上有較好的表現。




%在普適性方法中,我們用所提出的連線數機率分佈取代健全索靈頓分佈,同時保持隨機編碼過程不變,以維持LT碼的表現與通道狀況相互獨立。
%在給定限制條件下,藉由最小化目標函數,可得到相對應的連線數機率分佈,使其單一連線字符的平均數量,近似於事先決定的曲線,並使用序列二次規劃法(Sequential Quadratic Programming),尋找最小化問題的解。
%跟健全索靈頓分佈及過去的研究比較,我們所提出的連線數機率分佈,降低編解碼複雜度和完全解碼所需的字符平均數量。
%此外,我們在模擬中,使用基於802.11g WLAN的用戶數據報協議(User Datagram Protocol,UDP)封包丟失紀錄。
%相對應的結果顯示,跟健全索靈頓分佈比較,我們所提出的連線數機率分佈達到更高的傳輸效率,並分別降低至少$31.2\%$和$25.4\%$的編碼和解碼複雜度。



%本論文探討使用平行架構及無競爭式交錯器之渦輪碼解碼器來達到高速的資料輸出量。
%我們首先分析傳統平行架構的方法,整理各個方法的優缺點,也列出影響資料輸出量的關鍵因素。
%文中的討論主要是針對採用多個soft-in soft-out (SISO)處理器來對單一個接收到的字碼進行解碼這種技術所衍生出的議題。
%除了增加SISO處理器的數目之外,我們也結合了其他平行架構的方法來大幅提昇速度。
%然而,該方法將會導致相當高的硬體複雜度及降低處理器的運算效能。
%為了解決複雜度的問題,本論文介紹了兩種多層級的網路系統來負責平行解碼器中所有SISO處理器與記憶體之間的資料傳輸。
%這兩個網路系統分別支援採用inter-block permutation (IBP)交錯器及quadratic permutation polynomial (QPP)交錯器之渦輪碼解碼器。
%此多層級網路連結系統可以有效地降低實作時平行架構電路的繞線複雜度。
%至於另一個難題,則須透過調整處理器的執行程序來解決。
%我們提出兩套可防止資料相關性造成低運算效能的策略,並制定了各自對應的處理器執行程序。
%在這兩種高效能的程序中,其一是針對廣泛應用進行的設計,而另一則只能在特定的條件下使用。
%它們縮短了解碼流程中各個功能單元的閒置時間;因此,SISO處理器的運算效能得到了改善。
%
%
%基於上述之各項技術,我們實作了四個平行架構之渦輪碼解碼器。
%當中有兩個採用了IBP交錯器以及對應的多層級網路系統;它們皆使用了多個SISO處理器,且每個處理器在一個時脈週期中可處理複數筆資料;其中一個還利用了廣泛用途的高效能程序,讓硬體不會進入閒置的狀態。
%第三個解碼器則使用QPP交錯器和第二種網路連結系統;藉由這個裝置,再加上適當的控制電路,該解碼器最多可提供八個平行SISO處理器來支援在第三代合作伙伴計劃長期演進技術規格中所函括的全部區塊長度之渦輪碼解碼。
%最後一個解碼器亦使用QPP交錯器和多層級網路;它比其他三個解碼器有更高的平行度;另一方面,因為符合另一個高效能程序的限制條件,它也具備最高的處理器運算效能;這個解碼器的資料輸出量可達到$1.4$ Gb/s。
%實驗結果顯示我們所提的方法能夠得到預期的成效,也使得平行架構解碼器的速度有顯著的提昇。