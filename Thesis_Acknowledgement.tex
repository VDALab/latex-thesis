% !TEX root = Thesis.tex

在漫長的六年多的博士班的修業過程生涯即將劃下句點,如今下筆要寫出這段誌謝誠然百感交集,只因要感謝的人相當地多,難以一一細數。

首先感謝我親愛的父母以及家人,從我毅然決然下定決心從原本碩士班逕讀博士班之後,期間我的父母,潘宗昭先生以及陳秋菊女士。你們無怨無悔地支持,伴隨我走過這漫長的過程,是我從小至今終於完成學業最重要的基石。也感謝我的阿姨陳密教授以及姨丈顧鴻壽教授,感謝你們引導我走進這個學門,並且提供我在交大研修之時有個溫暖的住處。三十而立,我今終於能將這份博士論文獻給你們。
感謝我的指導老師:陳宏明教授。在我大學甫將畢業之際,徬徨無知該往何處從事研究時,是您對學生開放的觀念與用心,讓我願意在您的麾下學習至今。在我自碩士班研究直到博士班研究的修業過程中。承蒙陳宏明教授給予我最大的自由度從事有興趣的研究
,並且不斷地給予支持與建議。在我至德國研習期間也支持我在德的研究方向,並且促成我跟更多的研究人員互動發展。讓我在這塊領域獲得的不只是研究內容,更能瞭解研究倫理所在。承蒙陳宏明教授包容我不只一次的錯誤,讓我能夠從挫折中學習與成長。除此之外,陳宏明教授也鼓勵我不僅僅是學業上的發展,只要是能獲得成長的多面向發展也相當支持。在博士生涯的最後,您雖然人在美國訪問期間,仍然不辭辛勞地協助我完成博士論文與博士口試的流程。因為您的教導,才成就了今日的論文與我。

在交通大學VDA實驗室成長的過程,由衷感謝所有一起經歷過的實驗室同學們,包括與我一起入學的碩士班同學,一起經歷博士班的夥伴們還有實驗室伴隨我一起成長的學弟學妹們。感謝劉時穎同學、秦敬雨同學、王俊凱學長、蔡篤雄學長以及李杰叡同學,你們在研究上給我的策勵討論與合作,讓我在咱們實驗室的時光特別充實有趣。如果不是與你們一起討論研究的方向與實踐,偷懶的我恐怕沒這麼順利能夠做出這一點一滴的貢獻。同時也特別感謝陳巍仁教授在實驗電路上的給予支持,讓我的遷移設計研究能夠有實際上的實驗成果。博士班生涯時常得面對孤獨,而有一起努力的夥伴是難能可貴,感謝你們的伴隨,也祝福你們的美好未來。感謝赴台積電實習的主管以及同事們的協助,讓我能夠親澤最先進的技術並能夠在此學到非常多業界的方法。感謝在思源科技以及新思科技指導我遷移設計方面研究的陳東傑博士,您總是一針見血的看法與建議,讓我在研究過程中找到適切的方向前進。

特別感謝何宗易教授的牽線與指導教授陳宏明老師的協助,讓我能夠在博士修業的最後階段赴德國慕尼黑工業大學電子研究所訪問研究。在德國訪問修業期間,感謝慕尼黑工業大學的電子研究所長Prof. Dr.-Ing. Ulf Schlichtmann 以及我的Supervisor Dr.-Ing Helmut Graeb。兩位的指導與提攜讓我收穫非常多。也感謝所上的同學們特別是 Andreas Hermann與周邦彥同學,讓我在研究上或是在德生活上幫助不少。感謝訪德期間林柏宏教授對我的指導,讓我能夠順利完成期刊論文的精化。也萬分感激科技部與德國學術交流總署(DAAD)鼓勵國內博士研究生赴德研修,讓我們在研究的同時更俱備一個學術人才基本的國際觀與認識到自己的渺小。感謝台德三明治計畫期間的好夥伴,朱奕豪,林怡君與張雯琪同學的陪伴。最後特別感激鄭茹璘同學,在我赴德期間給予我的協助與支持鼓勵。妳讓我相信友情無國界,在世界的另一個角落也能遇到人生摯友如妳。由衷地相信妳也能順利完成博士學業,並且為妳的未來感到與有榮焉。

學生生涯至今告一個階段,但人生的學習仍未輕易結束,我想要親自表達感謝的人實在太多。一路走來,感謝國小老師梁淑娟女士的啟蒙與栽培,感謝國中老師劉劼珉先生對於以知識份子為己任的身教。沒有你們的啟發並且對我的栽培,我想我不會有如今的成就。感謝玩島玩的創業夥伴們,李嘉文與黃俊仁,包容我想要完成學業的任性至今。感謝新竹交通大學所給予我的一切,讓我能在這個環境裡被栽培成為博士。感謝中師實小的同學們,感謝中一中管樂團的夥伴們,感謝交大管樂團的朋友們,感謝AIESEC全世界的朋友們,以及三十年來我所認識人們的相遇與陪伴至今。與你們經歷的每一段經歷與回憶,都讓我深深感恩並造就了如今的我。僅將此份論文奉獻給你們,並願與你們一同共享這份榮耀。

\hspace{3em}

\begin{flushright}

潘柏丞\\
謹致\;\;於台灣新竹\\
二〇一五年五月二十一日
\end{flushright}




